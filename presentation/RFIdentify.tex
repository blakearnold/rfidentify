%
%
%
% 


\documentclass{article}
\usepackage{times, rfidentify}

	
	\title{RFIdentify: Speaker Presence Sensing}
	\author{Blake Arnold, Willi Ballenthin\\
	{\tt\{baa2121,wrb2102\}@columbia.edu}\\
	\\
	Internet Real-Time Lab, Columbia University\\
	New York, NY 10027\\
	% Place other locations here if you need them by uncommenting
	% the following lines.
	%\\
	%Atomic Computer \& Engineering Facility \\
	%Las Vegas NV, 034343\\
	}

	
	\reportmonth{Fall}
	\reportyear{2010}

\begin{document}

\makecover
\maketitle

\begin{abstract}
	
	Designing a location aware system to allow large conferences to work seamlessly has been a problem in the past. There have been
	 past try's at trying to create a location aware network and many successful implementations. But none of these successes has yet to allow an instantaneous notification of presence.

\end{abstract}


\section{Introduction} 
	
	Location aware hardware is slowly gaining ground in the past few years. The hardware 
	has continued to grow from Sonar to GPS to cellphone triangulation. 
	The main motivation for this technology is ease of use from the user side. New services
	 have been created to help third parties track users of such devices as means of recommendations or alerts.

	This technology can be useful during a large conference with numerous speakers.
	 Keeping track of which speakers are currently presenting is a daunting task. It requires
	  constantly looking at a time and the speaker order while also keeping track of any changes
	   to the speaker order.

	In this paper, we show how location aware hardware can be used to allow 
	better speaker recognition during a conference. Are main goal is to alert the 
	attendees when new speakers start to present. This allows the attendee to determine 
	when a speaker of interest begins his/her talk.

	Past designs to create a speaker aware system have been different in the past.


\section{Related Work}
	
	\begin{enumerate}
	\item {\bf F8 Conference} - 
	Facebook Presence for all attendees as well as speakers.

	\item {\bf Marc Petit-Huguenin} - 
	Original designer for microphone RFID system
	\end{enumerate}

\section{Outline}
	  In Section 4, we introduce the use cases of the new software and hardware combination. In Section 5 describes are strategy and design of the new system with further details in Section 6. Finally, we describe future work in Section 7."
\section{Problem}

\section{Approach}
	By using an RFID reader near the presenter location, the speaker can effectively be identified. The use of a RFID tag can raise security concerns
	from the users. RFID readers have been shown to be be able to read tags from few meters away.
	\subsection{Server}
	
	\begin{enumerate}
		
		\item	Dependencies
	\end{enumerate}
	
	\subsection{Client}
	
		\begin{enumerate}
		
			\item	Dependencies
			
				
			\item	Configure Handling
				
			\item	Avahi Loop
				
			\item	Reader Loop
		
		\end{enumerate}


\section{Technologies}
	The Client and server were implementing using open source software and other well known libraries to create a well tested code base.
	\subsection{Server}
	
		\begin{enumerate}
		\item	Apache
		Server core implementation for handling client requests. The httpcore.jar used in the implementation
		 handles low level functionality of client handling from opening ports to passing the requests onto higher level HTTP request handlers.
			
		\item	TextMarks
		
		\item SIMPLE
		SIP message passing implementation. Used to alert a presence server that a current speaker is presenting.
		
		
		
		\end{enumerate}

		

	\subsection{Client}
	
		\begin{enumerate}
		\item avahi/mDNS
		DESCRIPTION
		\item pthreads

		\item cURL
		DESCRIPTION
		\end{enumerate}


	\subsection{Hardware}
		\begin{enumerate}
		\item DLP

		\item FTDI/Big Reader

		\end{enumerate}
		
		

\section{Future Works}
	Create code for SenseEverything 

	
	\section{Summary and Future Work}
	often repeats the main result 
	\section{Acknowledgements}
	\section{Bibliography} 
	http://www.facebook.com/presence/

\end{document}
